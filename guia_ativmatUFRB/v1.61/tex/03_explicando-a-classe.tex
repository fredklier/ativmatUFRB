%==============================================================================
\section{Explicando a classe \texttt{ativmatUFRB.cls}}
%==============================================================================

Abaixo estão os comandos criados nessa classe para confecção das listas de 
atividade.

%------------------------------------------------------------------------------
\subsection{\texttt{\textbackslash titulo}}
%------------------------------------------------------------------------------
Esse comando deve ser colocado logo após o \verb|\begin{document}|.
Ele gera um cabeçalho estilizado, com logotipo da UFRB; e, é composto de seis 
itens obrigatórios, a saber:
\begin{enumerate}[(i)]
 \item \verb|\tituloDaLista{}|. Escreva o título da sua lista de atividade;
 \item \verb|\prof{}|. O nome do professor da disciplina;
 \item \verb|\disciplina{}|. A disciplina referente à lista, por exemplo,
  \emph{Funções de uma Variável Complexa}, \emph{Cálculo I}, etc.
 \item \verb|\curso{}|.  Curso onde está alocada a disciplina, por exemplo,
  \emph{Licenciatura em Matemática}, etc.
 \item \verb|\semestre{}|. Semestre onde se encontra a disciplina, por exemplo,
  $2^{\circ}$~semestre. Nesse caso, coloque \textbf{apenas} o numeral associado.
  No caso do exemplo citado, deverá ser feito assim: \verb|\semestre{2}|.
 \item \verb|\numeroDaLista{}|. O número da lista deve ser colocado em 
  \textbf{algarismos romanos}. 
  Por exemplo, se for a primeira lista, deve ser escrito assim:
  \verb|\numeroDaLista{I}|.
\end{enumerate}

Os seis itens citados acima devem ser colocados no preâmbulo do documento, ou
seja, entre o \verb|\documentclass{ativmatUFRB}| e o \verb|\begin{document}|.

\begin{codbox}{Exemplo de preenchimento do comando \texttt{\textbackslash titulo}}
\begin{verbatim}
\documentclass{ativmatUFRB}
\usandoXeLuaLaTeX %-> ou \usandopdfLaTeX
%=======================================
% Informações do Título da Lista
%=======================================
\tituloDaLista{Título da Lista}
\prof{Ícaro Vidal Freire}
\disciplina{Cálculo Vetorial e Integral}
\curso{Licenciatura em Matemática}
\semestre{2}
\numeroDaLista{I}
%=======================================
\begin{document}
\titulo %------> Comando para o cabeçalho
...
\end{document}
\end{verbatim}
\end{codbox}

%------------------------------------------------------------------------------
\subsection{Ambiente para Questões}
%------------------------------------------------------------------------------

O ambiente \verb|\begin{atividade} ...\end{atividade}| enumera uma lista com o
nome ``Questões'' estilizado: texto em negrito, dentro de uma caixa oval, sem
``indentação''.
Usamos assim:

\begin{tcblisting}
{
  title     = Ambiente para ``Questões'',
  fonttitle = \bfseries,
  colback   = blue!5!white,
  colframe  = blue!70!black,
  listing side text,
}
\begin{atividade}
 \questao Primeira questão.
 \questao Segunda questão.
\end{atividade}
\end{tcblisting}

Note que, para cada ``Questão estilizada'', usamos o comando \verb|\questao|.
Na Subseção~\ref{comand} falaremos sobre comandos para escrever dois tipos de 
alternativas nesse ambiente, cada uma com 5~(cinco) itens.

%------------------------------------------------------------------------------
\subsection{Ambiente para alternativas gerais}
%------------------------------------------------------------------------------

Os comandos \verb|\begin{itens}...\end{itens}| produz um ambiente propício para
os ``itens'' (alternativas) dentro do ambiente \texttt{atividade}.
Para produzir cada item, usamos o comando \verb|\item|.


\begin{tcblisting}
{
  title     = Ambiente geral para alternativas,
  fonttitle = \bfseries,
  colback   = blue!5!white,
  colframe  = blue!70!black,
  listing side text
}
\begin{atividade}
\questao Início da questão.
\begin{itens}
 \item Primeiro item.
 \item Segundo item.
\end{itens}
\end{atividade}
\end{tcblisting}

Você pode usar esse ambiente para produzir inúmeras listas, bastando para isso
colocar, entre \textit{colchetes}, o primeiro elemento da lista desejada.
A classe \texttt{ativmatUFRB.cls} fornece o comando \verb|\vf| para produzir um
espaço em branco entre dois \textit{parenteses}, \vf, para ser usado, por 
exemplo, em questões que envolvam ``\textcolor{red}{v}erdadeiro'' ou
``\textcolor{red}{f}also''. 

\begin{tcblisting}
{
  title     = Lista em romano (minúscula),
  fonttitle = \bfseries,
  colback   = blue!5!white,
  colframe  = blue!70!black,
  listing side text
}
\begin{atividade}
\questao Início da questão.
\begin{itens}[(i)]
 \item Primeiro item.
 \item Segundo item.
\end{itens}
\end{atividade}
\end{tcblisting}


\begin{tcblisting}
{
  title     = Lista em Romano (maiúscula),
  fonttitle = \bfseries,
  colback   = blue!5!white,
  colframe  = blue!70!black,
  listing side text
}
\begin{atividade}
\questao Início da questão.
\begin{itens}[(I)]
 \item Primeiro item.
 \item Segundo item.
\end{itens}
\end{atividade}
\end{tcblisting}

\begin{tcblisting}
{
  title     = Lista para ``verdadeiro'' ou ``falso'',
  fonttitle = \bfseries,
  colback   = blue!5!white,
  colframe  = blue!70!black,
  listing side text
}
\begin{atividade}
\questao Início da questão.
\begin{itens}[\vf]
 \item Primeiro item.
 \item Segundo item.
\end{itens}
\end{atividade}
\end{tcblisting}

%------------------------------------------------------------------------------
\subsection{Operadores matemáticos}
%------------------------------------------------------------------------------

Funções trigonométricas podem ser digitadas diretamente no idioma pt-BR.
Algumas funções foram omitidas: ou por já serem contempladas no idioma 
inglês\footnote{na realidade estão disponíveis nos pacotes da $\AmS$ 
(\textit{American Mathematical Society}).
Falaremos sobre ele na Seção~\ref{pacotes}, mas adiantamos que funções como
$\arg,\, \ln,\, \cos,$ etc., já se encontram no pacote citado.} (como 
\verb|\cos{}|, por exemplo), ou por não serem tão recorrentes (por exemplo,
\verb|\arccossec{}|).
Neste último caso, se for necessário, escreva no preâmbulo do seu texto o 
operador diretamente (no exemplo, acima exposto, pode ser obtido por:
\verb|\DeclareMathOperator{\arccossec}{arccossec}|).
A Tabela~\ref{tab:op} exibe os Operadores matemáticos disponíveis na classe
\href{https://ctan.dcc.uchile.cl/macros/latex/base/classes.pdf}{\texttt{ativmatUFRB.cls}}:
\marginpar{\qrcode[height = 1cm]{https://ctan.dcc.uchile.cl/macros/latex/base/classes.pdf}}

\begin{table}[!htbp]
 \centering
 \caption{Tabela com os Operadores da classe \texttt{ativmatUFRB.cls}}
 \label{tab:op}
 \begin{tabular}{lcl}
  \toprule
   \textbf{Operador} && \textbf{Saída}\\
  \midrule
   \verb|\sen|       && $\sen{}$\\
   \verb|\tg|        && $\tg$\\
   \verb|\cossec|    && $\cossec$\\
   \verb|\cotg|      && $\cotg$\\
   \verb|\arcsen|    && $\arcsen$\\
   \verb|\arctg|     && $\arctg$\\
   \verb|\arcsec|    && $\arcsec$\\
   \verb|\Ln|        && $\Ln$\\
   \verb|\Arg|       && $\Arg$\\
   \verb|\cis|       && $\cis$\\
  \bottomrule
\end{tabular}
\end{table}

%------------------------------------------------------------------------------
\subsection{\texttt{\textbackslash topico\{\}} \& Cia}
%------------------------------------------------------------------------------

Caso queira acrescentar tópicos, subtópicos ou até mesmo ``subsubtópicos'' 
entre blocos de questões, basta usar os respectivos comandos:
\verb|\topico{}|, \verb|\subtopico{}| ou \verb|\subsubtopico{}|.
Apenas o comando \verb|\topico{}| produz uma faixa cinza que engloba o título 
da seção desejada.
Os outros deixam as subseções e ``subsubseções'' em negrito com diminuição da
fonte (como na classe \verb|article.cls|).

%------------------------------------------------------------------------------
\subsection{Comandos úteis}\label{comand}
%------------------------------------------------------------------------------

A classe \texttt{ativmatUFRB.cls} disponibiliza alguns comandos úteis para 
construção de listas de atividade para matemática.
Obviamente os comandos não são exaustivos, visto que cada professor possui suas
particularidades nas disciplinas.
%
\subsubsection*{\textbackslash \texttt{vazio}} %-------------------------------
O comando \verb|\vazio| produz o símbolo do conjunto vazio, $\varnothing$.
%
\subsubsection*{\textbackslash \texttt{dd}} %----------------------------------
%
Digitando \verb|\dd| em um ambiente matemático produzirá ``$\dd$'', ou seja, a 
letra ``d'' usada como símbolo da diferencial (dica: para produzir um pequeno 
espaço antes desse comando use: ``\verb|\,|'', ou seja, \verb|\,\dd| ).
Por exemplo, numa integral em função da variável $x$, note a diferença sutil:

\begin{tcblisting}
{
  title     = $\dd$ vs $d$,
  fonttitle = \bfseries
  colback   = blue!5!white,
  colframe  = blue!70!black,
  listing side text, 
}
$\displaystyle \int f(x)\,\dd{x}$
\\
$\displaystyle \int f(x) dx$
\end{tcblisting}
%
\subsubsection*{\textbackslash \texttt{intc}} %--------------------------------
%
O simbolo para a integral de linha de uma curva fechada orientada no sentido 
anti-horário, $\displaystyle \intc$,  é dado pelo pacote \texttt{esint} através
do comando, não tão atrativo para a língua portuguesa, 
\verb|\varointctrclockwise|.
Na presente classe, o mesmo símbolo pode ser obtido usando \verb|\intc| (lembre
e ``\textcolor{red}{\textbf{int}}egral no sentido
\textcolor{red}{\textbf{c}}ontrário ao relógio'').
%
\subsubsection*{\textbackslash \texttt{versor}} %------------------------------
%
Suponha que você queira escrever algum vetor unitário em negrito e com o sinal 
do produto escalar usual, então basta usar o comando \verb|\versor{}|.

\begin{tcblisting}
{
  title     = Vetores unitários em negrito com o sinal do ``produto'',
  fonttitle = \bfseries,
  colback   = blue!5!white,
  colframe  = blue!70!black 
}
$\vv{F}(x,y,z)=L(x,y)\versor{i}+M(x,y)\versor{j}+N(x,y)\versor{k}$
\end{tcblisting}
%
\subsubsection*{\textbackslash \texttt{Resp}} %--------------------------------
%
Geralmente os alunos gostam de saber a resposta final de alguma questão para 
comparar com a resposta encontrada por eles.
Nesta classe, usamos o comando \verb|\Resp{}| para criar um ambiente com fonte 
menor, localizado sempre à direita, onde aparece o texto ``Resp.:'' em negrito.
Veja um exemplo abaixo:

\begin{tcblisting}
{
  title     = Resposta alinhada à direita,
  fonttitle = \bfseries,
  colback   = blue!5!white,
  colframe  = blue!70!black
}
Seja $\mathcal{D}$ a parte da coroa circular compreendida entre $x^2+y^2=1$ e $x^2+y^2=4$ no primeiro quadrante.
Calcule 
\[
  \iint\limits_{\mathcal{D}}(x^2+y^2)\,\dd{x}\,\dd{y}
\]
\Resp{$15\pi/8$}
\end{tcblisting}
%
\subsubsection*{\textbackslash \texttt{altercols}} %---------------------------
%
Para uma lista de alternativas, o ambiente padrão \texttt{itens} servirá 
adequadamente.
Entretanto, frequentemente, deseja-se construir uma lista com cinco alternativas
de resposta (onde uma está correta).
Para isso, usamos o comando \verb|\altercols{}{}{}{}{}{}| dentro do ambiente
\verb|\begin{atividade}|.
Perceba que esse comando possui 6~(seis) parâmetros: o primeiro é o número de 
colunas que pretende-se dividir as alternativas; e, os outros cinco parâmetros 
são as alternativas.
Por isso o nome do comando ``\textcolor{red}{\textbf{alter}}nativas em 
\textcolor{red}{\textbf{col}}una\textcolor{red}{\textbf{s}}''.
É obrigatório especificar os 6 (seis) parâmetros.
Obviamente, o primeiro parâmetro (número de colunas) varia de 1 a 5.

\begin{tcblisting}
{
  title     = Alternativas em 5 colunas,
  fonttitle = \bfseries,
  colback   = blue!5!white,
  colframe  = blue!70!black
}
\begin{atividade}
\questao Uma questão qualquer.
 \altercols{5}{0}{1}{2}{3}{4}
\end{atividade}
\end{tcblisting}

\begin{tcblisting}
{
  title     = Alternativas em 2 colunas,
  fonttitle = \bfseries,
  colback   = blue!5!white,
  colframe  = blue!70!black,
  listing side text
}
\begin{atividade}
\questao Uma questão qualquer.
 \altercols{2}{0}{1}{2}{3}{4}
\end{atividade}
\end{tcblisting}
%
\subsubsection*{\textbackslash \texttt{alterdce}} %----------------------------
%
Observe que no exemplo anterior, especificamente naquele em que se desejou 
escrever as alternativas em 2~(duas) colunas, houve um espaço vertical não tão 
conveniente entre as alternativas (d) e (e).
Para sanar esse problema, devemos usar o comando \verb|\alterdce{}{}{}{}{}|.
Note que só há 5~(cinco) parâmetros (todos dever ser preenchidos com as 
alternativas), pois não há variação no número de colunas (duas colunas fixas).
Assim, caso você queira ``\textcolor{red}{\textbf{alter}}nativas em
\textcolor{red}{\textbf{d}}uas \textcolor{red}{\textbf{c}}olunas 
\textcolor{red}{\textbf{e}}specíficas'', faça como no exemplo:

\begin{tcblisting}
{
  title     = Alternativas em 2 colunas específicas,
  fonttitle = \bfseries,
  colback   = blue!5!white,
  colframe  = blue!70!black
}
\begin{atividade}
\questao Uma questão qualquer.
 \alterdce{0}{1}{2}{3}{4}
\end{atividade}
\end{tcblisting}



